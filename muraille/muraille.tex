\textbf{TODO : Mettre les exos de 2016}

%%%%%%%%%%%% EXERCICES COLLEGE

\begin{exo}{\nstar{1}} Soit $l_1,l_2,\ldots,l_n$ des réels strictement positifs
tels qu'il existe un polygone dont les côtés sont de longueurs
respective $l_1,l_2,\ldots,l_n$. Montrer qu'il existe un polygone
convexe (avec éventuellement des angles plats) dont les côtés sont de
longueurs respective $l_1,l_2,\ldots,l_n$.
\end{exo}

%Matthieu Barré
\begin{exo}{}On se donne $2n+1$ nombres tels que la somme de $n$ nombres quelconques d'entre eux soit inférieure à la somme des $n+1$ autres. Montrer que tous ces nombres sont positifs.
\end{exo}

%%Matthieu Barré
\begin{exo}{}Soit $ABC$ un triangle équilatéral et $P$ un point à l'intérieur de ce triangle. Soient $D,E$ et $F$ les pieds des perpendiculaires de $P$ sur $[BC],[CA]$ et $[AB]$ respectivement. Montrer que
\begin{enumerate}
	\item $AF+BD+CE=AE+BF+CD$ et que
	\item $|APF|+|BPD|+|CPE|=|APE|+|BPF|+|CPD|$,
\end{enumerate}
où $|XYZ|$ désigne l'aire du triangle $XYZ$.
\end{exo}

% Nicolas
\begin{exo}{\nstar{1}}Déterminer tous les nombres réels $a$ et $b$ vérifiant l'égalité suivante :
$$2a^{2}+2b^{2}+2(b-a-ab)+2=0.$$
\end{exo}

%Gabriel
\begin{exo}{\nstar{0}}Sophie choisit au hasard un polynôme $P$
dont tous les coefficients sont des entiers naturels. Thomas a le
droit de demander à Sophie la valeur prise par $P$ en un nombre $a$
donné, puis, en tenant éventuellement compte de la première réponse
de Sophie, en un autre nombre $b$. Thomas doit ensuite deviner le
polynôme $P$. Comment faire ? 
\end{exo}

\begin{exo}{\nstar{2}}
Le nombre $10^{2013}$ est écrit au tableau. Alexandra et Béatrice jouent au jeu suivant à tour de rôle, où à chaque tour il est permis de faire l'une des deux opérations suivantes:
\begin{enumerate}
 \item[-] remplacer un nombre $x$ écrit au tableau par deux entiers $a,b>1$ tels que $ab=x$.
 \item[-] effacer un ou deux nombres égaux écrits au tableau.
  \end{enumerate}
Celle qui ne peut plus jouer a perdu. 

\smallskip

Alexandra joue en premier. Est-elle sure de gagner?
\end{exo}

\begin{exo}{\nstar{2}}
Montrer qu'il existe une infinité de nombres premiers dont le dernier chiffre n'est pas~$1$.
\end{exo}

%Matthieu Barré
\begin{exo}{\nstar{0}}Trouver tous les entiers relatifs $x$ et $y$ tels que \[x^2y+7x=x^2+3xy+3y+4\]
\end{exo}

\begin{exo}{\nstar{3}}
Soit $k$ un entier supérieur ou égal à $2$. Trouver tous les entiers $x$ et $y$ tels que: \[ y^k  = x^2 + x.\]
\end{exo}

\begin{exo}{\nstar{2}}Soit $[EF]$ un segment inclus dans le segment $[BC]$ tel que le demi-cercle de diamètre $[EF]$ est tangent à $[AB]$ en $Q$ et à $[AC]$ en $P$. Prouver que le point d'intersection $K$ des droites $(EP)$ et $(FQ)$ appartient à la hauteur issue de $A$ du triangle $ABC$.
\end{exo}

\begin{exo}{\nstar{3}}
Soit $ABCD$ un carr\'e, et $PQRS$ un petit carr\'e plac\'e \`a l'int\'erieur de $ABCD$ de telle sorte que les segments $[AP]$, $[BQ]$, $[CR]$, $[DS]$ ne s'intersectent pas entre eux, et n'intersectent pas $PQRS$. Prouver que la somme des aires des quadrilat\`eres $ABQP$ et $CDSR$ est \'egale \`a la somme des aires des quadrilat\`eres $BCRQ$ et $DAPS$.
\end{exo}

\begin{exo}{\mbox{ }$^ {\star}$}
Trouver tous les entiers $a$, $b$ et $m$ tels que \[4a^b + 1 = m^2.\]
\end{exo}

%Nicolas
\begin{exo}{}Soit $ABC$ un triangle. Soit $I$ un point du segment $[AB]$. Montrer que la droite $(CI)$ est la bissectrice intérieure issue de $C$ si et seulement si $CA/CB=IA/IB$.
\end{exo}

\begin{exo}{\mbox{ }$^ {\star}$}
Montrer que $\sqrt{2}+\sqrt{5}+\sqrt{13}$ est irrationnel.
\end{exo}

%Gabriel
\begin{exo}{\nstar{0}}Montrer que tous les termes de la suite
$\left(a_{n}\right)$ définie par 
\[
\begin{cases}
a_{1}=a_{2}=a_{3}=1\\
a_{n+1}=\frac{1+a_{n-1}a_{n}}{a_{n-2}}
\end{cases}
\]
sont des entiers. 
\end{exo}

\begin{exo}{ \mbox{ }$^ {\star \star}$}
On a un polyèdre convexe, dont les faces sont des triangles. Les sommets
du polyèdre sont coloriés
avec trois couleurs.

\smallskip

Montrer que le nombre de triangles dont les sommets ont trois couleurs
distinctes est pair.
\end{exo}

%Gabriel
\begin{exo}{\nstar{0}}
 Existe-t-il une
suite $\left(u_{n}\right)$ d'entiers naturels non nuls telle que
$u_{n}$ et $u_{m}$ sont premiers entre eux ssi $\left|n-m\right|=1$
?
\end{exo}

%Gabriel
\begin{exo}{\nstar{0}}  Alexandre et Béatrice jouent au jeu suivant
: sur un rectangle quadrillé de taille $4\times2015$, chacun place
à son tour un pentomino $\mathbf{T}$, qui ne recouvre pas les précédents.
Le premier joueur qui ne peut plus poser de pentomino a perdu. Montrer
que le premier joueur à commencer possède une stratégie gagnante.
\end{exo}

%Guillaume
\begin{exo}{\nstar{1}}Trouver l'ensemble des entiers naturels égaux au carré de leur somme des chiffres. 
\end{exo}

%Guillaume
\begin{exo}{\nstar{1}}Trouver les entiers naturels $n$ tels que $n^4+4^n$ soit premier. 
\end{exo}

%Matthieu Barré
\begin{exo}{\nstar{0}}Montrer que pour tous réels strictement positifs $a$ et $b$ et pour tout entier $n$,
\[\left(1+\frac{a}{b}\right)^n+\left(1+\frac{b}{a}\right)^n \geq 2^{n+1}.\]
\end{exo}

\begin{exo}{\nstar{2}}
Trouver les entiers strictement positifs $a$ et $b$ tels que
$\frac{a^{2b+1} b-1}{a+1}$ et $\frac{b^a a+1}{b-1}$ soient des
entiers.
\end{exo}

\begin{exo}{\nstar{2}}
On considère $ab+1$ condylures (avec $a,b$ des entiers positifs) telles que si l'on considère $2$ quelconques d'entre eux, alors soit l'un descend de l'autre, soit ils n'ont aucune lien de parenté. Montrer que l'on peut en trouver $a+1$ qui descendent les uns des autres ou bien $b+1$ qui n'ont aucun lien de parenté entre eux.
\end{exo}


\begin{exo}{\nstar{2}}
Soient $m,n$ des entiers positifs. À quelle condition existe-t-il un entier positif $N$
tel que pour tout entier $r \geq N$ il existe deux entiers positifs $a$ et $b$ tels que  $r=am+bn$ ? Estimer la valeur minimale de $N$ aussi précisément que possible.
\end{exo}

%Gabriel
\begin{exo}{\nstar{0}}Montrer que $2015$ n'est pas une somme
de trois cubes d'entiers naturels. 
\end{exo}

\begin{exo}{\nstar{3}}Soit $x_0, x_1, x_2, \ldots $ une suite de nombres réels. On dit que la suite $x_0, x_1, \ldots$ est convexe si :
\[
\textrm {pour tout entier } n \geq 0, \frac{x_{n-1}+x_{n+1}}{2}\geq x_n
\]
et qu'elle est log-convexe si :
\[
\textrm {pour tout entier } n \geq 0,  x_{n+1}x_{n-1}\geq x_n^2
\]
 On suppose que pour tout nombre réel $a>0$, la suite $x_0, a x_1, a^2 x_2^2, a^3 x_3^3, \ldots$ est convexe. Prouver que la suite $x_0, x_1, \ldots $ est log-convexe.

\end{exo}

%Gabriel
\begin{exo}{\nstar{0}}Un troupeau est constitué de $25$ vaches
pesant chacune entre $500$ et $1000$ kg. Montrer qu'il existe dans
ce troupeau deux sous-troupeaux contenant au moins une vache, et sans
aucune vache en commun, dont la masse est égale au gramme près. 
\end{exo}

\begin{exo}{\nstar{3}}Soit $ABC$ un triangle isocèle en $C$. On considère un point $P$ sur le cercle circonscrit au triangle $ABC$ situé entre $A$ et $B$ (et $P$ n'est pas du même côté que $C$ par rapport à la droite $(AB)$). Soit $D$ un point de la droite $(PB)$ tel que les droites $(CD)$ et $(PB)$ soient perpendiculaires. Prouver que $PA+PB= 2 \cdot PD$.
\end{exo}

\begin{exo}{\nstar{3}}Soient $x,y>0$, et soit $s=min(x,y+1/x,1/y)$. Quelle est la valeur maximale possible de $s$ ? Pour quels $x,y$ est-elle atteinte ?

\end{exo}

\begin{exo}{ \nstar{3}}
Trouver les nombres premiers $p$ pour lesquels il existe des entiers positifs $x$ et $y$ tels que:
$$
x(y^2-p)+y(x^2-p)=5p.
$$\end{exo}

\begin{exo}{\nstar{4}}
On place $2n$ points dans le plans et on trace $n^2+1$ segments entre ces points. 

Montrer que l'on peut trouver 3 points reli\'es deux \`a deux.
\end{exo}

\begin{exo}{\nstar{3}}
Soit $\mathcal{P}=A_1\ldots A_{2n}$ un $2n-$gone convexe dans le plan. Soit $P$ un point intérieur à $\mathcal{P}$, non situé sur une diagonale. Prouver que $P$ est contenu dans un nombre pair de triangles à sommets parmi les $A_i$.
\end{exo}

%Guillaume
\begin{exo}{\nstar{1}}En joignant chaque milieu d'un côté d'un rectangle d'aire $1$ aux sommets du côté opposé, on obtient un octogone au centre du rectangle (une figure est conseillée). Quelle est son aire ? 
\end{exo}

%Guillaume
\begin{exo}{\nstar{1}}
Trouver tous les entiers strictement positifs $x,y,z$ tels que $$\frac{1}{x}+\frac{2}{y}-\frac{3}{z}=1.$$
\end{exo}


%Guillaume
\begin{exo}{\nstar{1}}On demande à Guillaume d'additionner deux fractions irréductibles. Hélas, l'étourdi les multiplie. Heureusement, le résultat est le même : une fraction dont le dénominateur est $2007$. Quel peut être le numérateur?
\end{exo}

%Guillaume
\begin{exo}{\nstar{1}} Trouvez $7$ entiers strictement positifs, tous différents, tels que chacun divise leur somme, et que cette somme soit la plus petite possible.
\end{exo}


\begin{exo}{\nstar{3}}
Soit $ABC$ un triangle isocèle en $A$ tel que $\widehat{BAC}<60^{\circ}$. Les points $D$ et $E$ sont des points du côté $[AC]$ tels que $EB=ED$ et $\widehat{ABD}=\widehat{CBE}$. Soit $O$ le point d'intersection des bissectrices des angles $\widehat{BDC}$ and $\widehat{ACB}$. Trouver la valeur de l'angle $\widehat{COD}$.
\end{exo}

\begin{exo}{\nstar{3}}Trouver tous les entiers positifs $a,b,c$ tels que $$ 2^{a}3^{b}+9 = c^{2}.$$
\end{exo} 

\begin{exo}{\nstar{3}} Soit $ABC$ un triangle dont tous les angles sont aigus. On note $ \Gamma$ son cercle circonscrit. On suppose que La tangente en $A$ à $\Gamma$ coupe la droite $(BC)$ en un point $P$. Soit $M$ le milieu de $[AP]$ et soit $R$ le deuxième point d'intersection de la droite $(BM)$ avec le cercle $ \Gamma$. La droite $(PR)$ recoupe le cercle $ \Gamma$ en $S$.

Prouver que les droites $(AP)$ et $(CS)$ sont parallèles.
\end{exo}


%Arsène
\begin{exo}{\nstar{1}} 
On considère $2014$ tas de jetons, le $i$-ème tas contenant $p_{i}$ jetons, où $p_{i}$ est le
$i$-eme nombre premier (notamment $p_{1} = 2$). On a le droit de fusionner deux
piles puis d’ajouter $1$ jeton a la nouvelle pile, ou de diviser une pile en deux
autres (pas forcément de meme taille), puis d’ajouter $1$ jeton à l’une des deux
nouvelles piles crées. Pourra-t-on avoir $2014$ piles de $2014^{2014}$ jetons ?
\end{exo}

%Nicolas
\begin{exo}{\nstar{1}}Soient $ABC$ un triangle équilatéral et $M$ un point situé à l’intérieur de ce triangle. Montrer que la
somme des distances de $M$ aux trois côtés du triangle est indépendante de $M$.
\end{exo}


%Margaret
\begin{exo}{\nstar{1}} 
Alceste et Brunehilde jouent au jeu suivant: on commence par dessiner un carré de côté 1. Ensuite les joueurs jouent tour à tour, sachant que chaque coup consiste à dessiner un carré ne se superposant pas à la figure déjà dessinée, et tel que l'un de ses côtés coïncide exactement avec l'un des côtés de la figure déjà dessinée (en particulier, cette dernière est toujours un rectangle). Le gagnant est celui qui arrive à une figure d'aire un multiple de 5. On suppose qu'Alceste commence. L'un des deux joueurs a-t-il une stratégie gagnante? 
\end{exo}



%%%%%%%%% Exos seconde-premiére %%%%%%%%%%%

%Matthieu Barré
\begin{exo}{\nstar{0}}Existe-t-il un triangle rectangle ayant des côtés de longueur rationnelle et dont l'aire vaut 1 ?
\end{exo}

%Matthieu Barré
\begin{exo}{ \nstar{0}} Soit $\mathcal{P}$ la parabole d'équation $y=x^2$. Trouver une condition nécessaire et suffisante pour que des points de $\mathcal{P}$ d'abscisse $a$, $b$, $c$ et $d$ soient cocycliques.
\end{exo}

%Guillaume
\begin{exo}{\nstar{1}} Pour quels entiers $n \geq 1$ existe-t-il une bijection $\sigma : \{ 1,2, \ldots,n\} \rightarrow \{ 1,2, \ldots,n\}  $ de sorte que $\vert \sigma(i)-i \vert \neq \vert \sigma(j)-j \vert$ si $i \neq j$?
\end{exo}

\begin{exo}{\nstar{2}}
Consid\'erons un triangle $ABC$ tel que $AB=BC$ et $\widehat{ABC} = 80^\circ$. Soit $P$ le point de l'int\'erieur de $ABC$ tel que $\widehat{PAC} = 40^\circ$ et $\widehat{PCA} = 30^\circ$. Calculer l'angle $\widehat{BPC}$.
\end{exo}

\begin{exo}{ \nstar{3}}Soit $n>2$ un entier naturel, $T$ la transformation $\mathbb{R}^n\rightarrow \mathbb{R}^n,(x_1,\ldots,x_n)\mapsto (\frac{x_1+x_2}{2},\frac{x_2+x_3}{2},\ldots,\frac{x_{n-1}+x_{n}}{2},\frac{x_n+x_1}{2})$. On part d'un $n-uplet$ $(a_1,\ldots,a_n)$ d'entiers deux à deux distincts. Prouver que pour $k\in \mathbb{N}$ assez grand, $T^k(a_1,\ldots,a_n)\notin \mathbb{N}^n$.
\end{exo}

\begin{exo}{\nstar{2}}
On considère un triangle acutangle $ABC$ avec $\widehat{A}=60^\circ$. Déterminer les points $M\in (AB)$ et $N\in (AC)$ qui minimisent la somme $|CM|+|MN|+|BN|$.
\end{exo}

%Gabriel
\begin{exo}{\nstar{0}}Soit $P$ un point de l'espace
et $r>0$. Montrer qu'il existe $8$ sphères disjointes de même rayon
$r$ qui cachent le point $P$, c'est-à-dire que toute demi-droite
issue de $P$ rencontre au moins l'une des sphères. On supposera que
les centres des sphères sont tous à des distances $>r$ de $P$.

\end{exo}

%Gabriel
\begin{exo}{\nstar{0}}Parmi les quadrilatères de côtés
$a$,$b$,$c$,$d$, caractériser géométriquement celui qui a la plus
grande aire.
\end{exo}

\begin{exo}{ \nstar{3}}
Soit $P$ un point à l'intérieur d'un polygone régulier à $n$ côtés tel que quel que soit un côté du polygone, $P$ se projette orthogonalement sur l'intérieur du côté. Ces $n$ projections orthogonales partagent les $n$ côtés du polygone en $2n$ segments. On numérote ces segments de $1$ à $2n$, en commençant par un segment arbitraire, puis en tournant dans le sens direct le long du polygone. Prouver que la somme des longueurs des segments de numéros pairs est égale à la somme des longueurs des segments de numéros impairs.
\end{exo}

%Matthieu Barré
\begin{exo}{\nstar{0}}Soit $P$ un point à l'intérieur d'un cercle de rayon $R$, $d$ une droite passant par $P$ et $d'$ la perpendiculaire à $d$ en $P$. On fait tourner les droites $d$ et $d'$ d'un angle $\phi$ autour de $P$. Montrer que quelle que soit la position du point $P$, l'aire balayée par $d$ et $d'$ à l'intérieur du cercle (qui a la forme d'une croix) vaudra $\pi R^2 \frac{4\phi}{360}$.
\end{exo}

%Guillaume
\begin{exo}{\nstar{1}}Trouver toutes les fonctions $f : \R \rightarrow \R$ telles que pour tous réels $x,y,z,t$,
$$(f(x)+f(y))(f(z)+f(t))=f(xz+yt)+f(xt-yz).$$
\end{exo}

\begin{exo}{\nstar{2}}
Soit $ABC$ un triangle, $D$ le milieu de $[AB]$, $M$ le milieu de $[CD]$. On suppose
que les droites $(BM)$ et $(AC)$ sont sécantes, en un point que l'on nomme $N$.
Enfin, soit $\Gamma$ le cercle circonscrit au triangle $BCN$.
Montrer que $AB$ est tangente à $\Gamma$ si et seulement si $$\frac{BM}{MN}
= \frac{(BC)^2}{(BN)^2}.$$
\end{exo}


\begin{exo}{\nstar{2}}
Soit $n$ un entier relatif. Montrer qu'il existe un unique polynôme $P$ à coefficients
dans $\lbrace 0, 1 \rbrace$  tel que $P(-2)=n$.
\end{exo}

\begin{exo}{\nstar{3}}
Si $A$ est un ensemble d'entiers naturels non vide, on note $PGCD(A)$ le plus grand diviseur commun \`a tous les \'el\'ements de $A$.

Trouver le plus petit entier naturel non nul $n$ v\'erifiant la propri\'et\'e suivante : pour toute partie $A$ de $\{1, \, 2, \, ..., \, 2012\}$, il existe une partie $B$ de $A$ de cardinal inf\'erieur ou \'egal \`a $n$ telle que $PGCD(B) = PGCD(A)$.
\end{exo}



%Gabriel
\begin{exo}{\nstar{0}}Alice et Bob sont complices et opposés à
Eve. Dans une salle est disposé un échiquier 
avec sur chaque case une pièce de monnaie en position pile ou face.
Au départ, seuls Eve et Bob sont dans la pièce ; ce dernier observe
Eve retourner l'une des pièces. Ensuite, Eve s'en va et Bob a le droit
de retourner au plus une pièce avant de partir lui aussi. Finalement,
Alice entre dans la salle. Elle observe l'échiquier et doit déterminer
la pièce retournée par Eve. Expliquer comment Alice et Bob peuvent
se concerter à l'avance pour qu'Alice retrouve à coup sûr la pièce
retournée par Eve.
\end{exo}


%Guillaume
\begin{exo}{\nstar{1}}Soit $ABC$ un triangle rectangle isocèle en $C$, $D$ et $E$ des points de $[AC]$ et $[BC]$ tels que $CE=CD$. Les perpendiculaires à $(AE)$ passant par $D$ et $C$ coupent $(AB)$ respectivement en $K$ et $L$. Montrer que $KL=LB$.
\end{exo}


\begin{exo}{\nstar{1}}Démontrer que pour tout nombre réel $x \in ]0, \pi/2[$ on a l'inégalité suivante:
$$ \left( 1+ \frac{1}{\cos^{10}(x)} \right)  \left( 1+ \frac{1}{\sin^{10}(x)} \right) \geq 1089.$$
\end{exo}

%Gabriel
\begin{exo}{\nstar{0}}On considère $n$ entiers $a_{1},\ldots a_{n}$
dans $\left\{ 1,\ldots2015\right\} $ tels que le ppcm de deux d'entre
eux est toujours $>2015$. Montrer que 
\[
\frac{1}{a_{1}}+\cdots+\frac{1}{a_{n}}<2.
\]
\end{exo}

%Margaret
\begin{exo}{\nstar{1}}Déterminer tous les couples de polynômes non constants $P$ et $Q$ unitaires, de degré $n$ et admettant $n$ racines positives ou nulles (non nécessairement distinctes) tels que
$$P(x) - Q(x) = 1.$$
\end{exo}

\begin{exo}{\nstar{2}}
On appelle $I$ l'ensemble des points du plan tels que leur abscisse et leur ordonnée soient des nombres irrationnels, et $R$ celui des points dont les deux coordonnées sont rationnelles. Combien de points de $R$ au maximum peuvent se situer sur un cercle de rayon irrationnel dont le centre appartient à $I$ ? 
\end{exo}

%Margaret
\begin{exo}{\nstar{1}}On se donne un nombre entier $n>1$. Deux joueurs $R$ et $B$ colorient tour à tour des points sur un cercle, $R$ coloriant en rouge et $B$ en bleu. Une fois que chacun a placé $n$ points, le jeu s'arrête, et chaque joueur cherche sur le cercle l'arc de cercle le plus long ayant pour extrémités des points de sa couleur, et ne contenant aucun autre point coloré. Le joueur dont l'arc de cercle sélectionné est le plus long gagne (s'ils sont de même longueur, ou bien s'il n'y a aucun tel arc, on dit que la partie est nulle). L'un des deux joueurs a-t-il une stratégie gagnante?
\end{exo}

%Margaret
\begin{exo}{\nstar{1}}Pour chaque nombre premier $p$, trouver le plus grand entier $k$ tel que $(p!)^k$ divise $(p^2)!$.
\end{exo}

%Margaret
\begin{exo}{\nstar{1}}Soit $a_1,a_2,\ldots$ une suite infinie strictement croissante d'entiers naturels telle que pour tout $n$, le terme $a_n$ soit égal soit à la moyenne arithmétique, soit à la moyenne géométrique des deux termes $a_{n-1}$ et $a_{n+1}$. Cette suite est-elle nécessairement toujours arithmétique ou toujours géométrique à partir d'un certain rang?
\end{exo}

%Margaret
\begin{exo}{\nstar{1}}On partitionne un carré en un nombre fini (supérieur à 2) de rectangles de côtés parallèles aux côtés du carré. Parmi les segments joignant les centres de deux rectangles de la partition, en existe-t-il toujours un n'intersectant aucun autre rectangle?
\end{exo}


\begin{exo}{\nstar{3}}Soient $ABC$ un triangle, $H$ le pied de la hauteur issue de $B$, $M$  le milieu de $[AB]$ et $N$ le milieu de $[BC]$. Les cercles circonscrits aux triangles $AHN$ et $CHM$ se recoupent au point $P$. Prouver que la droite $(PH)$ coupe le segment $[MN]$ en son milieu.
\end{exo}

\begin{exo}{\nstar{2}}
On considère trois nombres réels $x,y,z$ qui ne sont pas tous égaux. On suppose que 
$$x+ \frac{1}{y}= y+ \frac{1}{z}=z+ \frac{1}{x}=k.$$
Trouver toutes les valeurs possible de $k$.
\end{exo}



%Gabriel
\begin{exo}{\nstar{0}}Soient $a$, $b$ et $c$ les longueurs
des côté d'un triangle. Montrer que 

\[
a^{2}b\left(a-b\right)+b^{2}c\left(b-c\right)+c^{2}a\left(c-a\right)\geqslant0
\]
Quel est le cas d'égalité ? 
\end{exo}

%Matthieu Barré
\begin{exo}{\nstar{0}}
Soit $ABC$ un triangle équilatéral de côté $a$ et $P$ un point à l'intérieur de ce triangle. On construit un triangle $XYZ$ de côtés de longueur $PA$, $PB$ et $PC$ et on note $F$ son point de Fermat. Montrer que $FX+FY+FZ=a$.
\end{exo}

\begin{exo}{\nstar{2}}
Une droite passant par un point $A$ coupe un cercle $ \mathcal{C}$ en $B$ et $C$. On suppose que $B$ est situé entre $A$ et $C$. Les deux tangentes à $ \mathcal{C}$ passant par $A$ sont tangentes à $ \mathcal{C}$ en $S$ et en $T$. On note $P$ le point d'intersection de $(ST)$ et $(AC)$. Montrer que 
$ \frac{AP}{PC}=2\frac{AB}{BC} $

\end{exo}

%Guillaume
\begin{exo}{\nstar{1}}Trouver toutes les fonctions $f$ de $ \N$ dans $ \N$ telles que pour tous entiers $m,n \geq 0$ on ait
$$f(m+f(n))=f(f(m))+f(n).$$
\end{exo}

\begin{exo}{\nstar{3}}
Une ampoule est placée sur chaque case d'un échiquier $2011\times 2012$. Initialement, $4042111$ ampoules sont allumées. On a le droit d'éteindre une ampoule si elle appartient à un bloc $2 \times 2$ dont les trois autres ampoules sont éteintes. Peut-on éteindre toutes les ampoules?
\end{exo}



\begin{exo}{\nstar{2}}
Soit $ABC$ un triangle dont le cercle inscrit est noté $ \omega$. Soient $I$ le centre  de $ \omega$ et $P$ un point tel que les droites $(PI)$ et $(BC)$ soient perpendiculaires et les droites $(PA)$ et $(BC)$ parallèles. Soient finalement $Q$ et $R$ deux points tels que $Q \in (AB)$, $R \in (AC)$, les droites $(QR)$ et $(BC)$ soient parallèles et finalement $(QR)$ soit tangente à $ \omega$.

Prouver que $ \widehat {QPB}= \widehat {CPR}$.
\end{exo}

%Matthieu Barré
\begin{exo}{\nstar{0}}
Montrer que pour tous entiers $m$ et $n$ non nuls et strictement positifs,
\[\frac{1}{\sqrt[n]{1+m}}+\frac{1}{\sqrt[m]{1+n}}\geq 1\]
\end{exo}

\begin{exo}{\nstar{3}}
 Soit $p$ un nombre premier congru à 2 modulo 3 et $a$ et $b$ des entiers tels que $p$ divise $a^2 + ab + b^2$. Montrer que $p$ divise $a$ et $b$.\end{exo}


\begin{exo}{\nstar{2}}
Trouver tous les polynômes $P(x)$ à coefficients réels tels que 
$$x P \left( \frac{y}{x} \right)+y P \left( \frac{x}{y} \right)=x+y$$
pour tous nombres réels non nuls $x,y$.
\end{exo}

\begin{exo}{\nstar{3}}
Soit $n$ un entier non nul. Montrer que \[ \lfloor \sqrt{n} \rfloor + \lfloor \sqrt[3]{n} \rfloor + \ldots + \lfloor \sqrt[n]{n} \rfloor = \lfloor \log_2 n \rfloor + \lfloor \log_3 n \rfloor + \ldots + \lfloor \log_n n \rfloor.\]
On rappelle que $\lfloor x \rfloor$ désigne le plus grand entier inférieur ou égal à $x$, et que $\lfloor \log_k n \rfloor$ est égal au plus grand exposant $a$ tel que $k^a \leq n$.
\end{exo}


\begin{exo}{\nstar{2}}
Trouver tous les nombres premiers $p,q$ tels que $ p^2-pq-q^3=1 $.
\end{exo}

%Nicolas
\begin{exo}{\nstar{1}} Trouver tous les polynômes $P$ à coefficients réels tels que $P(0)=0$ et tels que $P(X^{2}+1)=P(X)^{2}+1$.
\end{exo}

%%Matthieu Barré
\begin{exo}{\nstar{0}}Trouver tous les polynômes $P$ à coefficients réels tels que, pour tout $n \in \mathbb{N}$, il existe un rationnel $r$ tel que $P(r)=n$.
\end{exo}

%Nicolas
\begin{exo}{\nstar{1}}Soit $(x_{n})_{n \geq 0}$ une suite de nombre réels telle que pour tout entier positif $n \geq 0$ on ait
$$ \sum_{i=0}^{n}x_{i}^{3}= \left( \sum_{i=0}^{n}x_{i} \right) ^{2}.$$
Montrer que pour tout entier $n \geq 0$, il existe un entier $m \geq 0$ tel que
$$ \sum_{i=0}^{n}x_{i}= \frac{m(m+1)}{2}.$$
\end{exo}

%%Matthieu Barré
\begin{exo}{\nstar{0}}
Montrer que pour tous réels strictement positifs $x,y$ et $z$,
\[\frac{x}{x+2y+3z}+\frac{y}{y+2z+3x}+\frac{z}{z+2x+3y}\geq \frac{1}{2}\]
\end{exo}

\begin{exo}{\nstar{2}}
Trouver tous les couples $(p,n)$, où $p$ est  un nombre premier et $n$ un entier strictement positif, tels que $p^n$ divise $(p-1)! +1$.
\end{exo}

%Guillaume
\begin{exo}{\nstar{1}}On définit une suite $u_{n}$ ainsi : $u_{1}$ et $u_{2}$ sont des entiers entre $1$ et $10000$ (au sens large), et $u_{k+1}$
est la plus petite valeur absolue des différences deux à deux des termes précédents. Montrer que
$u_{21} = 0$.
\end{exo}

%Matthieu Barré
\begin{exo}{\nstar{0}} Soit $ABC$ un triangle acutangle, avec $AC > BC$. On note $H$ son orthocentre, $O$ le centre de son cercle circonscrit et $M$ le milieu de $[AC]$. Soit $F$ le pied de la hauteur issue de $C$, et $P$ le symétrique de $A$ par rapport à $F$. On note $X$ l'intersection de $(PH)$ avec $(BC)$, $Y$ l'intersection de $(FX)$ avec $(OM)$, et $Z$ l'intersection de $(OF)$ avec $(AC)$. Montrer que $F, M, Y$ et $Z$ sont cocycliques.

\end{exo}

%%Matthieu Barré
\begin{exo}{\nstar{0}}Soit $P$ un polynôme à coefficients entiers. Existe-t-il des entiers distincts $a$, $b$ et $c$ tels que
\begin{align*}
P(a)=b\\
P(b)=c\\
P(c)=a
\end{align*}
\end{exo}


%Guillaume
\begin{exo}{\nstar{1}}Hyacinthe et Hippolyte jouent au jeu suivant : sur un échiquier $m \times n$, on place une tour sur
une case $c$. Tour à tour, chacun la déplace (d’un nombre arbitraire de cases selon les lignes ou
colonnes, comme aux échecs). On perd lorsqu’on est obligé de revenir sur une case où la tour
s’est déjà arrêtée. Qui gagne ?
\end{exo}

\begin{exo}{\nstar{2}}
On considère un cercle $ \mathcal{C}_{1}$ du plan, d'équation $(x-4)^{2}+y^{2}=1$, ainsi qu'une droite $l$ de pente positive qui passe par l'origine et qui est tangente à $ \mathcal{C}_{1}$ en $P_{1}$. Le cercle $ \mathcal{C}_{2}$ est tangent à l'axe des abscisse en $P_{2}$, passe par $P_{1}$ et son centre appartient à $l$.

\smallskip

Le cercle $ \mathcal{C}_{3}$ est tangent à $l$ en $P_{3}$, passe par $P_{2}$ et son centre appartient à l'axe des abscisses. Le cercle $ \mathcal{C}_{4}$ est tangent à l'axe des abscisses en $P_{4}$, passe par $P_{3}$ et son centre appartient à $l$.

\smallskip

On construit de la même manière les cercles $ \mathcal{C}_{5}, \mathcal{C}_{6}$, etc. Pour $n \geq 1$, on note $S_{n}$ l'aire du cercle $ \mathcal{C}_{n}$.

\smallskip
Lorsque $N$ tend vers l'infini, vers quoi converge la somme
$$ \sum_{n=1}^{N} S_{n} \qquad ?$$
\end{exo}


\begin{exo}{ \mbox{ }$^ {\star \star}$}Soit $n\geq 0$ un entier naturel. Montrer qu'il existe un disque dans le plan contenant exactement $n$ points à coordonnées entières.
\end{exo}

%Guillaume
\begin{exo}{\nstar{1}}Al et Xandre communiquent via un réseau peu fiable : lorsqu’Al envoie un message de $n$ caractères,
Xandre reçoit $k$ d’entre eux (dans le même ordre). Sachant que tant que certaines
sous-suites du message original ne sont pas sorties, le réseau ne renvoie pas une sous-suite déjà
obtenue par Xandre, combien de fois Al doit-il envoyer son message (dont tous les caractères sont
distincts) pour être sûr que Xandre puisse le décoder ?
\end{exo}

%Guillaume
\begin{exo}{\nstar{1}}
Soit un ensemble de $n$ points $(a_{i}, b_{i})$ dans le carré $[0, 1] \times  [0, 1]$, les $a_{i}$ différant deux à deux, les
$b_{i}$ aussi, et contenant les points $(0,0)$ et $(1, 1)$. Sissi la suave sauterelle veut aller du premier au
second, et s’astreint à respecter la règle suivante : si elle est en $(a_{i}, b_{i})$ , elle peut sauter en $(a_{j}, b_{j})$ si et seulement si :

-$a_{i} < a_{j}$, $b_{i} < b_{j}$,

-il n’y a aucun $a_{k}$ entre $a_{i}$ et $a_{j}$ ou aucun $b_{k}$ entre $b_{i}$ et $b_{j}$ .

\bigskip

Pour la piéger, Arabelle l’araignée acharnée décide de mettre en place une configuration où un
tel trajet est impossible. Quelle est la valeur minimale de $n$ pour qu’elle puisse arriver à ses fins
?
\end{exo}

%http://www.artofproblemsolving.com/Forum/viewtopic.php?f=57&t=553283
%http://www.artofproblemsolving.com/Forum/viewtopic.php?p=3551877&sid=350159384ed78145a0f56d979fd43774#p3551877
\begin{exo}{\nstar{1}} 
Soit $n \geq 4$ un entier. On considère des entiers strictement positifs $a_{1}, \ldots,a_{n}$ placés sur un cercle. On suppose que chaque terme $a_{i}$ ($1 \leq i \leq n$) divise la somme de ses deux voisins, c'est-à-dire qu'il existe un entier $k_{i}$ tel que 
\[ \frac{a_{i-1}+a_{i+1}}{a_i}= k_i \],
avec la convention $a_{0}=a_{n}$ et $a_{n+1}=a_{1}$. Montrer que
$$2n \leq k_{1}+ k_{2}+ \cdots k_{n} <3n.$$
 \end{exo}

%Matthieu Barré
\begin{exo}{\nstar{0}}Soit $n \in \mathbb{N}$. Montrer que si $2+2\sqrt{28n^2+1}$ est un entier, alors c'est un carré parfait.
\end{exo}

%Matthieu Barré
\begin{exo}{\nstar{0}}On écrit un nombre premier $p=a_ka_{k-1}...a_0$ en base 10 et on pose
\[Q_p(x)=a_kx^k+a_{k-1}x^{k-1}+...+a_1x+a_0\]
Montrer que $Q_p$ n'a pas de racines entières sauf pour 4 valeurs de $p$ que l'on déterminera.
\end{exo}

%%Matthieu Barré
\begin{exo}{\nstar{0}}Soit $P$ un polynôme de degré 2015 tel que $P(0)=0$, $P(1)=1$, ..., $P(2014)=2014$ et $P(2015)=2016$. Combien vaut $P(2016)$ ?
\end{exo}


%\pagebreak

%EXOS AVANCES

\begin{exo}{\nstar{0}}Montrer que les droites rouges sur la figure ci-dessous sont concourantes (on part de 5 points du plan formant un pentagone convexe) :
\end{exo}

\begin{exo}{ \nstar{0}}Soit $ABC$ un triangle d'orthocentre $H$. On prend un point $P$ sur $[BC]$ et on appelle $D$ le projeté orthogonal de $H$ sur $[AP]$. On trace la parallèle à $(BC)$ passant par $D$ : elle recoupe $(ABà$ en $E$, $(AC)$ en $F$, le cercle circonscrit à $ADB$ en $X$ et le cercle circonscrit à $ADC$ en $Y$. Enfin, soit $Z$ l'intersection de $(XB)$ et $(YC)$. Montrer que $ZE=ZF$ si et seulement si $P$ est le milieu de $[BC]$.
\end{exo}

%Vincent
\begin{exo}{\nstar{0}}Soit $ABC$ un triangle, $D$, $E$, $F$ les points de tangence du cercle inscrit aux côtés $[BC]$, $[CA]$, $[AB]$. Soit $\Delta$ la parallèle à $(BC)$ (différente de $(BC)$) tangente au cercle inscrit. La droite $\Delta$ intersecte $(AB)$ et $(AC)$ respectivement en $P$ et $Q$. Soit $T$ l'intersection de $(BC)$ et $(EF)$, et $M$ le milieu de $[PQ]$. Montrer que $(TM)$ est tangente au cercle inscrit. 
\end {exo}

%Vincent B.
\begin{exo}{\nstar{0}}Montrer qu'il existe une infinité d'entiers positifs $n$ tels que $n^2 + 1$ n'ait que $1$ et lui-même comme diviseur de la forme $k^2 + 1$. 
\end{exo}

%Guillaume
\begin{exo}{\nstar{1}}Soit $u_{n}$ le nombre de résidus différents, modulo $n$, des entiers de la forme $k(k+1)/2$ avec $k \geq 0$. Calculer $u_{n}$.
\end{exo}

%Margaret
\begin{exo}{\nstar{1}}Il y a $2014$ députés dans une assemblée. Chacun d'eux déteste exactement trois autres députés, sachant que le fait de détester n'est pas nécessairement réciproque: $A$ peut détester $B$ sans que $B$ déteste $A$. Quel est le plus petit $n$ tel que l'on puisse repartir les 2014 députés en $n$ comités, de sorte qu'aucun député ne se retrouve dans le même comité avec quelqu'un qu'il déteste?
\end{exo}

\begin{exo}{\nstar{2}}
Soit $ABC$ un triangle dont le cercle circonscrit est noté $ \omega$, le centre du cercle inscrit $ \omega_{1}$ est noté $I$ et le centre du cercle exinscrit $ \omega_{2}$ tangent au côté $[BC]$ est noté $I_{A}$. Les cercles $ \omega_{1}$ et $ \omega_{2}$ sont tangents à $[BC]$ respectivement en $D$ et en $E$. Soit finalement $M$ le milieu de l'arc $ \wideparen {BC}$ qui ne contient pas $A$. On considère un cercle tangent à $ \omega$ en un point $T$ et à la droite $(BC)$ en $D$. Soit $S$ l'intersection  de $(TI)$ avec $ \Omega$. Prouver que les droites $(SI_{A})$ et $ (ME)$ se coupent sur $ \omega$.
\end{exo}

\begin{exo}{\nstar{2}}
Soient $a,b,c$ des réels strictement positifs tels que
$$a+b+c=a^{1/7}+b^{1/7}+c^{1/7}.$$
Prouver que $$a^{a} b^{b} c^{c} \geq 1.$$
\end{exo}


\begin{exo}{  \nstar{4}}
Alice et Bob jouent sur un \'{e}chiquier plan infini. Alice commence par
choisir une case et la colorie en rouge, puis Bob choisit une case non
encore colori\'{e}e et la colorie en vert, et ainsi de suite. Alice gagne la
partie si elle r\'{e}ussi \`{a} colorier en rouge quatre cases  dont les
centres forment les sommets d'un carr\'{e} de c\^{o}t\'{e}s parall\`{e}les
\`{a} ceux des cases.

a) Prouver qu'Alice poss\`{e}de une strat\'{e}gie gagnante.

b) Qu'en est-il si Bob a, lui, le droit de colorier deux cases en vert \`{a}
chaque coup?
\end{exo}

%Vincent B.
\begin{exo}{\nstar{0}}Soit $ABC$ un triangle et $N$ son point de Nagel. Soit $\Delta$ une droite qui passe par $N$. La droite $\Delta$ recoupe $(BC)$, $(CA)$ et $(AB)$ en $D$, $E$ et $F$ respectivement. Soit $X$ le symétrique de $D$ par rapport au milieu de $[BC]$. On définit de même $Y$ et $Z$. 
\begin{enumerate}
\item[(i)] Montrer que $X$, $Y$ et $Z$ sont alignés.
\item[(ii)] Montrer que la droite passant par XYZ est tangente au cercle inscrit de $ABC$.
\end{enumerate}
\end{exo}

%Vincent
\begin{exo}{\nstar{1}}Soit $n$ un nombre parfait, c'est-à-dire un entier tel que $$\sum_{d
\mid n}d = 2n.$$
On factorise $n$ sous la forme d'un produit de nombres premiers : $$n =
\prod_{i=1}^k p_i^{\alpha_i},$$ avec $p_1 < p_2 < \ldots < p_k$.
Montrer que $\alpha_1$ est pair.
\end{exo}


%Matthieu Barré
\begin{exo}{\nstar{0}}On place quatre points dans le plan, et on suppose que les six distances qui les relient entre eux sont toutes entières. Montrer qu'alors au moins une d'entre elles est divisible par 3.
\end{exo}

%http://www.artofproblemsolving.com/Forum/viewtopic.php?p=2745851&sid=e3ddfde67499106e5685d25928204743#p2745851
\begin{exo}{\nstar{2}}
Soit $ABCD$ un quadrilatère tel que $AC=BD$. Soit $P$ le point d'intersection des diagonales $(AC)$ et $(BD)$. On note $\omega_{1}$ le cercle circonscrit de $ABP$. Soit $O_{1}$ le centre de $ \omega_{1}$. On note $ \omega_{2}$ le cercle circonscrit de $CDP$. Soit $O_{2}$ le centre de $ \omega_{2}$. On note $S$ et $T$ les intersections respectives de $ \omega_{1}$ et $ \omega_{2}$ avec $[BC]$ (autres que $B$ et $C$). Soient $M$ et $N$ les milieux respectifs des arcs $ \wideparen {SP}$ (ne contenant pas $B$) et $ \wideparen {TP}$ (ne contenant pas $C$). Prouver que les droites $(MN)$ et $(O_{1} O_{2})$ sont parallèles.
\end{exo}

%Vincent B.
\begin{exo}{\nstar{0}}Soit $ABC$ un triangle. Soit $K$ l'intersection des tangentes au cercle circonscrit de $ABC$ en $B$ et en $C$, et soit $A'$ le second point d'intersection de $(AX)$ avec le cercle circonscrit de $ABC$. On définit similairement  $B'$ et $C'$ de manière cyclique. Soit $P$ est un point quelconque. On définit les secondes intersections de $(AP)$, $(BP)$ et $(CP)$ avec le cercle circonscrit de $ABC$ comme étant $A''$, $B''$ et $C''$. Soit $X$ l'intersection de $(BC)$ et de $(A'A'')$, et on définit similairement $Y$ et $Z$ de manière cyclique. Montrer que $X$, $Y$ et $Z$ sont alignés.
\end{exo}

\begin{exo}{\nstar{4}}
L'ensemble $\lbrace 1, 2, \ldots ,3n\rbrace$ est partitionné en trois ensembles $A$, $B$ et $C$ de $n$ éléments chacun.

 Montrer qu'il est possible de choisir un élément dans chacun de ces trois ensembles, tels que la somme de deux d'entre eux soit égale au troisième.
\end{exo}

%Matthieu Barré
\begin{exo}{\nstar{0}}Soient $a$, $b$ et $c$ les côtés entiers et premiers entre eux d'un triangle rectangle. Montrer que si $c$ est un carré parfait, alors l'aire du triangle est divisible par 84.
\end{exo}

\begin{exo}{\nstar{5}}
Les \'el\`eves d'une classe sont all\'es se chercher des glaces par groupe d'au moins deux personnes. Il y a eu $k>1$ groupes en tout. Deux élèves quelconques sont partis ensemble exactement une fois.

 Prouver qu'il n'y a pas plus de $k$ \'el\`eves dans la classe.
\end{exo}

%Vincent
\begin{exo}{\nstar{1}}Soit $m$ et $n$ deux entiers naturels non nuls. On note $\phi(m,n)$ le
cardinal de l'ensemble $\{k : 1 \leq k \leq n, \mathrm{PGCD}(k,m) =
1\}$. Trouver tous les entiers $m \geq 1$ tels que $n \phi(m,m) \leq m
\phi(m,n)$ pour tout $n \in \mathbb{N}^\ast$.
\end{exo}


\begin{exo}{\nstar{2}}
Trouver toutes les fonctions $f : \R_{+} \rightarrow \R_{+}$ telles que pour tous $a>b>c>d>0$ vérifiant $ad=bc$ on ait
$$f(a+d)+f(b-c)=f(a-d)+f(b+c).$$

\end{exo}

%Matthieu Barré
\begin{exo}{\nstar{0}}Trouver tous les entiers $x$ et $y$ tels que \[y^2=x^3+(x+4)^2.\]
\end{exo}

\begin{exo}{\nstar{3}}
On prend un quadrilatère convexe $ABCD$, et on divise chacun de ses côtés en $N$ portions égales. On relie ensuite ces différentes portions de façon à former un quadrillage $N\times N$, comme sur la figure. On sélectionne ensuite $N$ quadrilatères de ce quadrillage, de telle sorte qu'il n'y en ait pas deux sur la même ligne ou sur la même colonne du quadrillage. Calculer l'aire totale de tous ces quadrilatères.

\begin{center}
\shorthandoff{:;!?}
\begin{tikzpicture}[line cap=round,line join=round,>=triangle 45,x=0.4cm,y=0.4cm,scale=1.5]
\clip(-2.12,-3.22) rectangle (8.84,5.32);
\fill[fill=black,fill opacity=0.1] (-0.2,1.48) -- (0.6,2.01) -- (0.94,1.43) -- (0.18,0.95) -- cycle;
\fill[fill=black,fill opacity=0.1] (3.48,2.12) -- (4.21,2.55) -- (4,3.35) -- (3.24,2.87) -- cycle;
\fill[fill=black,fill opacity=0.1] (1.49,-1.88) -- (2.09,-1.68) -- (2.47,-2.2) -- (1.91,-2.35) -- cycle;
\fill[fill=black,fill opacity=0.1] (1.29,0.85) -- (2.02,1.27) -- (2.34,0.64) -- (1.64,0.27) -- cycle;
\fill[fill=black,fill opacity=0.1] (3.28,-1.27) -- (3.88,-1.07) -- (3.6,-0.38) -- (2.97,-0.63) -- cycle;
\fill[fill=black,fill opacity=0.1] (6.56,-2.08) -- (6.03,-2.17) -- (5.85,-1.32) -- (6.41,-1.17) -- cycle;
\fill[fill=black,fill opacity=0.1] (3.32,0.31) -- (3.98,0.63) -- (4.23,-0.12) -- (3.6,-0.38) -- cycle;
\fill[fill=black,fill opacity=0.1] (4.64,0.94) -- (5.31,1.26) -- (5.13,2.11) -- (4.43,1.74) -- cycle;
\draw (-1,0.94)-- (5.38,5.22);
\draw (5.38,5.22)-- (6.56,-2.08);
\draw (6.56,-2.08)-- (2.32,-2.82);
\draw (2.32,-2.82)-- (-1,0.94);
\draw (-0.2,1.48)-- (2.85,-2.73);
\draw (3.38,-2.64)-- (0.6,2.01);
\draw (1.39,2.55)-- (3.91,-2.54);
\draw (4.44,-2.45)-- (2.19,3.08);
\draw (2.99,3.62)-- (4.97,-2.36);
\draw (5.5,-2.27)-- (3.79,4.15);
\draw (4.58,4.69)-- (6.03,-2.17);
\draw (6.41,-1.17)-- (1.91,-2.35);
\draw (1.49,-1.88)-- (6.27,-0.26);
\draw (6.12,0.66)-- (1.08,-1.41);
\draw (0.66,-0.94)-- (5.97,1.57);
\draw (5.82,2.48)-- (0.25,-0.47);
\draw (-0.17,0)-- (5.68,3.4);
\draw (5.53,4.31)-- (-0.59,0.47);
\draw (-0.2,1.48)-- (0.6,2.01);
\draw (0.6,2.01)-- (0.94,1.43);
\draw (0.94,1.43)-- (0.18,0.95);
\draw (0.18,0.95)-- (-0.2,1.48);
\draw (3.48,2.12)-- (4.21,2.55);
\draw (4.21,2.55)-- (4,3.35);
\draw (4,3.35)-- (3.24,2.87);
\draw (3.24,2.87)-- (3.48,2.12);
\draw (1.49,-1.88)-- (2.09,-1.68);
\draw (2.09,-1.68)-- (2.47,-2.2);
\draw (2.47,-2.2)-- (1.91,-2.35);
\draw (1.91,-2.35)-- (1.49,-1.88);
\draw (1.29,0.85)-- (2.02,1.27);
\draw (2.02,1.27)-- (2.34,0.64);
\draw (2.34,0.64)-- (1.64,0.27);
\draw (1.64,0.27)-- (1.29,0.85);
\draw (3.28,-1.27)-- (3.88,-1.07);
\draw (3.88,-1.07)-- (3.6,-0.38);
\draw (3.6,-0.38)-- (2.97,-0.63);
\draw (2.97,-0.63)-- (3.28,-1.27);
\draw (6.56,-2.08)-- (6.03,-2.17);
\draw (6.03,-2.17)-- (5.85,-1.32);
\draw (5.85,-1.32)-- (6.41,-1.17);
\draw (6.41,-1.17)-- (6.56,-2.08);
\draw (3.32,0.31)-- (3.98,0.63);
\draw (3.98,0.63)-- (4.23,-0.12);
\draw (4.23,-0.12)-- (3.6,-0.38);
\draw (3.6,-0.38)-- (3.32,0.31);
\draw (4.64,0.94)-- (5.31,1.26);
\draw (5.31,1.26)-- (5.13,2.11);
\draw (5.13,2.11)-- (4.43,1.74);
\draw (4.43,1.74)-- (4.64,0.94);
\end{tikzpicture}
\shorthandon{:;!?}
\end{center}
\end{exo}



\begin{exo}{\nstar{4}}
Une suite croissante $(s_n) _ { n \geq 0}$ est dite super-additive si pour tout couple $(i,j)$ d'entiers on a $s_{i+j} \geq s_i + s_j$. Soient $(s_n)$ et $(t_n)$ deux telles suites super-additives. Soit $(u_n)$ la suite croissante d'entiers vérifiant qu'un nombre apparaît autant de fois dans $(u_n)$ que dans $(s_n)$ et $(t_n)$ combinées. 

Montrer que $(u_n)$ est elle aussi super-additive.
\end{exo}

\begin{exo}{\nstar{4}}
Soit $ABC$ un triangle, soit $A'B'C'$ un triangle directement semblable \`a ABC de telle sorte que $A$ appartienne au c\^ot\'e $B'C'$, $B$ au c\^ot\'e $C'A'$ et $C$ au c\^ot\'e $A'B'$. Soit $O$ le centre du cercle circonscrit \`a $ABC$, $H$ son orthocentre et $H'$ celui de $A'B'C'$. 

Montrer qu'on a $OH=OH'$.
\end{exo}

%%Matthieu Barré 
\begin{exo}{\nstar{0}} Trouver le plus petit réel $M$ tel que \[|(a-b)(a-c)(b-c)(a+b+c)| \leq M(a^2+b^2+c^2)^2\]
pour tous nombres réels $a,b,c$.
\end{exo}

%Margaret
\begin{exo}{\nstar{1}} Soit $k\geq 6$ un entier  et $P$ un polynôme à coefficients entiers tel qu'il existe $k$ entiers distincts $x_1,\ldots,x_k$ tels que pour tout $i\in \{1,\ldots,k\}$, $P(x_i) \in \{1,\ldots,k-1\}$. Montrer que $P(x_1) = \ldots = P(x_k)$. 
\end{exo}

%Guillaume
\begin{exo}{\nstar{1}}Soient $k \geq 2$ un entier et $a \geq k-1$ un nombre réel. Montrer que pour tout $n$-uplet de nombres réels strictement positifs $(x_{1}, \ldots, x_{n})$ on a
$$ \frac{x_{1}+ \cdots +x_{n}}{1+a} \leq  \frac{x_{1}^{k+1}}{x_{1}^{k}+a x_{2}^{k}}+ \cdots+ \frac{x_{n}^{k+1}}{x_{n}^{k}+a x_{1}^{k}}.$$
\end{exo}

%Guilaume
\begin{exo}{\nstar{1}}On considère une ligne de $n$ carrés. On note $S(n)$ le nombre minimal de carrés à colorier en bleu
tels que chacun des $n-1$ traits séparant deux cases voisines soit à égale distance de deux cases
bleues. Montrer que
$$ \lfloor 2 \sqrt {n-1} \rfloor+1 \leq S(n) \leq  \lfloor 2 \sqrt {n} \rfloor +1.$$
\end{exo}


%Matthieu Barré
\begin{exo}{{\nstar{0}}}Soit $n \geq 3$ un nombre entier et $x_{1}, \ldots, x_{n}$ des nombres réels strictement positifs. Montrer que \[\frac{x_1}{x_2+x_3}+\frac{x_2}{x_3+x_4}+...+\frac{x_n}{x_1+x_2}\geq \frac{5}{12}.\]
\end{exo}

%http://www.artofproblemsolving.com/Forum/viewtopic.php?p=3551902&sid=350159384ed78145a0f56d979fd43774#p3551902
\begin{exo}{\nstar{1}}Soit $k \geq 1$ un entier. Trouver tous les polynômes $P$ à coefficients entiers tels que $P(n)$ divise $(n!)^{k}$ pour tout entier $n \geq 1$.
\end{exo}

%http://www.artofproblemsolving.com/Forum/viewtopic.php?p=7970&sid=350159384ed78145a0f56d979fd43774#p7970
\begin{exo}{\nstar{1}} On dit qu'une permutation  $a_{1}, \ldots, a_{n}$ des entiers $1,2 \ldots,n$ est \emph{sympathique} s'il existe au moins un carré parfait parmi les entiers $ a_{1}, a_{1}+a_{2}, \ldots, a_{1}+a_{2}+ \cdots+a_{n}$. Trouver tous les entiers $n$ tels que toutes les permutations de $1,2 \ldots,n$ soient sympatiques.
\end{exo}

\begin{exo}{\nstar{4}}
On consid\`ere un ensemble de $2n+1$ droites du plan, deux jamais parall\`eles ni perpendiculaires et trois jamais concourantes. Trois droites forment donc toujours un  triangle non-rectangle.

D\'eterminer le nombre maximal de triangles aigus qui peuvent ainsi \^etre form\'es.
\end{exo}

\begin{exo}{\nstar{2}}
Soit $n$ un entier. Dans les lignes d'un tableau de $2^n$ lignes et $n$ colonnes on place tous les $n$-uplets form\'es de $1$ et de $-1$. Ensuite, on efface certains de ces nombres, et on les remplace par des $0$. Prouver que l'on peut trouver un ensemble de lignes dont la somme est nulle (i.e., tel que, pour tout $i$, la somme des nombres appartenant \`a la colonne $i$ d'une ligne de notre ensemble soit nulle).
\end{exo}


%Vincent B.
\begin{exo}{\nstar{0}}Soit $X$ un point à l'intérieur du triangle $ABC$ tel que $XA.BC = XB.CA = XC.AB$. Soient $I_1, I_{2}, I_{3}$  les centre des cercles inscrits respectifs de $BXC$, $AXC$ et $AXB$. Montrer que les droites $AI_1$, $BI_2$ et $CI_3$ sont concourantes.


\end{exo}

\begin{exo}{\nstar{3}}Soit $ABC$ un triangle, et $O$ un point à l'intérieur de ce triangle. Quel est le point $M$ minimisant la quantité $AM+BM+CM+OM$?
\end{exo}

%http://www.artofproblemsolving.com/Forum/viewtopic.php?p=2429835&sid=350159384ed78145a0f56d979fd43774#p2429835
\begin{exo}{\nstar{1}}Soient $a,b,c$ des nombres réels strictement positifs tels que $a+b+c=3$. Prouver que
$$ \frac{a}{1+(b+c)^2}+\frac{b}{1+(a+c)^2}+\frac{c}{1+(a+b)^2}\le\frac{3(a^2+b^2+c^2)}{a^2+b^2+c^2+12abc}.$$
\end{exo}

%http://www.artofproblemsolving.com/Forum/viewtopic.php?p=3024260&sid=350159384ed78145a0f56d979fd43774#p3024260
\begin{exo}{\nstar{1}}Soit $ABC$ un triangle non isocèle. Son cercle inscrit, de centre $I$, touche le côté $[BC]$ en $D$. Soit $X$ un point de l'arc $ \wideparen {BC}$ du cercle circonscrit de $ABC$ tel que si $E,F$ sont respectivement les projetés orthogonaux de $X$ sur $(BI)$ et $(CI)$ et $M$ le milieu de $[EF]$, alors $MB=MC$. Prouver que $ \widehat{BAD}=\widehat{CAX} $.
\end{exo}


\begin{exo}{\nstar{5}}
Soit $E$ un ensemble de $n \geq 2$ points du plan. On d\'esigne respectivement par $D$ et $d$ la plus grande et la plus petite distance entre deux points distincts de $E$. 

Prouver que:
$$D \geq  \frac{ \sqrt {3}}{2} ( \sqrt {n}-1) d.$$
\end{exo}


\begin{exo}{\nstar{4}}
Soit $S$ un ensemble infini de points du plan tel que si $A,B$ et $C$ sont trois points quelconques  dans $S$, la distance de $A$ \`a la droite $(BC)$ soit un entier.

Prouver que les points de $S$ sont tous align\'es.
\end{exo}

\begin{exo}{\nstar{3}}
Soient $a,b,c>0$ des nombres réels tels que $a+b+c=3$. Prouver que:
$$ \frac{ab}{b^{3}+1}+\frac{bc}{c^{3}+1}+\frac{ca}{a^{3}+1}\le\frac{3}{2}.$$
\end{exo}


\begin{exo}{\nstar{5}}
Trouver toutes les fonctions continues $f: \C \rightarrow \C$ v\'erifiant:
$$f(x+y) f(x-y) = f(x)^2-f(y)^2.$$
\end{exo}

%Vincent B.
\begin{exo}{\nstar{0}}Soit $ABC$ un triangle, $I$ son centre du cercle inscrit. Soient $D$, $E$ et $F$ ses projetés sur les côtés de $ABC$. Soient $P$ et $Q$ les intersections de la droite $(EF)$ avec le cercle circonscrit de $ABC$. Soient $O_1$ et $O_2$ les centres des cercles circonscrits de $AIB$ et de $AIC$. Montrer que le centre du cercle circonscrit de $DPQ$ est sur la droite $(O_1O_2)$.
\end{exo}

\begin{exo}{\nstar{2}}
Soient $P,Q$ deux polynômes non nuls à coefficients entiers tels que $ \deg P>\deg Q$. On suppose que le polynôme $p \cdot P+Q$ possède une racine rationnelle pour une infinité de nombre premiers $p$. Montrer qu'au moins une racine de $P$ est rationnelle.
\end{exo}

\begin{exo}{\nstar{3}}Soient $P$ et $Q$ deux polyn\^omes \`a coefficients entiers, premiers entre eux. Pour tout entier $n$, posons $u_n = PGCD(P(n), \, Q(n))$. Montrer que la suite $(u_n)$ est p\'eriodique.
\end{exo}




\begin{exo}{\nstar{4}}
On veut colorier certains des points de l'ensemble $E_{n}=\{(a,b)/a,b$
entiers et $0\leq a,b\leq n\}$ de sorte que tout carr\'{e} $k\times k$ dont
les sommets sont dans $E_{n}$ contienne au moins un point colori\'{e} sur
son bord. On note $m(n)$ le nombre minimum de points \`{a} colorier pour que
la condition d\'{e}sir\'{e}e soit satisfaite.

Prouver que $${\lim_ {{n\rightarrow +\infty }} }\frac{m(n)}{n^{2}}=%
\frac{2}{7}.$$
\end{exo}

%http://www.artofproblemsolving.com/Forum/viewtopic.php?p=2667975&sid=350159384ed78145a0f56d979fd43774#p2667975
\begin{exo}{\nstar{1}}Soit $ \Gamma$ le cercle circonscrit d'un triangle acutangle $ABC$. Le point $D$ est le centre de l'arc $ \wideparen {BC}$  contenant  $A$, et $I$ est le centre du cercle inscrit de $ABC$. La droite $(DI)$ coupe $(BC)$ en $E$ et recoupe $ \Gamma$ en $F$. Soit $P$ un point de la droite $(AF)$ tel que $(PE)$ et $(AI)$ soient parallèles. Prouver que $(PE)$ est la bissectrice de l'angle $ \widehat {BPC}$.
\end{exo}

%Vincent B.
\begin{exo}{\nstar{0}}Soit $ABC$ un triangle. On note $D,E,F$ les points de tangence du cercle inscrit de centre $I$ de $ABC$ avec les côtés de $ABC$. Soit $X$ le point de $[AB]$ tel que $(XD)$ et $(EF)$ soient perpendiculaires. Soit $Y$ le second point d'intersection des cercles $AEF$ et $ABC$. Montrer que le triangle $XYF$ est rectangle.
\end{exo}