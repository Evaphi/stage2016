\textbf{TODO : Mettre à jour 2016 !}

Une muraille de 142 exercices était affichée dans la bibliothèque - la plus petite des cinq salles à notre disposition. Les exercices 1 à 42 sont de niveau 1, 43 à 96 de niveau 2, 97 à 142 de niveau 3. Un exercice est décoré de n étoiles lorsqu'il est resté sans solution à la muraille de $n$ stages.

Les élèves du groupe A cherchent les exercices de niveau 1 (ou au dessus). Les élèves du groupe B, les exercices de niveau 2 et les exercices étoilés de niveau 1 (ou au dessus). Les élèves du groupe C, ceux de niveau 3 et ceux étoilés de niveau 2 (ou au dessus). Les élèves du groupe D, ceux de niveau 3. Certains exercices ont été résolus avant la constitution des groupes lundi soir.

Une fois un exercice résolu, la solution doit etre rédigée et donnée à une animatrice ou un animateur. La première solution correcte d'un exercice est reproduite dans ce polycopié. Il est possible de résoudre les exercices à plusieurs. Le but est d'avoir tout résolu à la fin du stage !

Tous les deux exercices résolus (individuellement ou bien par équipe d'au plus deux personnes), une glace est offerte. La distribution des glaces a été retardée car les premiers jours nous n'avions pas accès au frigidaire. A la fin du stage, quatre Grands Prix Mystère seront décernés aux quatre élèves ayant obtenu le plus de points en résolvant des exercices de la muraille, dans chacun des quatre groupes A, B, C, D. Un autre Grand Prix Mystère est décerné à l'équipe (constituée d'au moins deux élèves et d'au plus quatre élèves) ayant obtenu le plus de points en résolvant des exercices de la muraille.

Barème : un exercice à $x$ étoiles résolu rapporte $x+1$ points (sauf pour les élèves du groupe B qui résolvent des exercices étoilés de niveau 1 et les élèves du groupe C qui résolvent des exercices étoilés de niveau 2, pour lesquels un exercice à $x$ étoiles rapporte $x$ points).

