\begin{sol}[99](Résolu par Baptiste Serraille)

		\definecolor{uuuuuu}{rgb}{0.26666666666666666,0.26666666666666666,0.26666666666666666}
		\definecolor{qqqqff}{rgb}{0.,0.,1.}
		\begin{center}
		\begin{tikzpicture}[line cap=round,line join=round,>=triangle 45,x=0.5cm,y=0.5cm]
		\clip(-4.8866410764292,-5.572904036609516) rectangle (24.92335892357077,7.055095963390474);
		\draw (3.38,4.66)-- (4.6,-3.1);
		\draw (4.6,-3.1)-- (13.6,-3.2);
		\draw (3.38,4.66)-- (13.6,-3.2);
		\draw(6.6073797978803945,-0.7786041073342618) circle (1.171777726594992cm);
		\draw (3.8617737962416823,1.595602738659465)-- (7.415842784909827,1.556113083229819);
		\draw (1.0600044164045799,-3.0606667157378284)-- (5.638808290575755,1.575857910944642);
		\draw (1.0600044164045799,-3.0606667157378284)-- (6.581341825154507,-3.1220149091683833);
		\draw (1.0600044164045799,-3.0606667157378284)-- (8.036095081092272,1.0790893016257088);
		\begin{scriptsize}
		\draw [color=qqqqff] (3.38,4.66)-- ++(-2.5pt,-2.5pt) -- ++(5.0pt,5.0pt) ++(-5.0pt,0) -- ++(5.0pt,-5.0pt);
		\draw[color=qqqqff] (3.5393589235707914,5.053095963390476) node {$A$};
		\draw [color=qqqqff] (4.6,-3.1)-- ++(-2.5pt,-2.5pt) -- ++(5.0pt,5.0pt) ++(-5.0pt,0) -- ++(5.0pt,-5.0pt);
		\draw[color=qqqqff] (4.199358923570791,-2.712904036609518) node {$B$};
		\draw [color=qqqqff] (13.6,-3.2)-- ++(-2.5pt,-2.5pt) -- ++(5.0pt,5.0pt) ++(-5.0pt,0) -- ++(5.0pt,-5.0pt);
		\draw[color=qqqqff] (13.747358923570783,-2.800904036609518) node {$C$};
		\draw [color=uuuuuu] (8.036095081092272,1.0790893016257088)-- ++(-2.5pt,-2.5pt) -- ++(5.0pt,5.0pt) ++(-5.0pt,0) -- ++(5.0pt,-5.0pt);
		\draw[color=uuuuuu] (8.445358923570787,1.3350959633904789) node {$E$};
		\draw [color=uuuuuu] (4.292261123652807,-1.1425789504473618)-- ++(-2.5pt,-2.5pt) -- ++(5.0pt,5.0pt) ++(-5.0pt,0) -- ++(5.0pt,-5.0pt);
		\draw[color=uuuuuu] (3.913358923570791,-0.9309040366095195) node {$F$};
		\draw [color=uuuuuu] (6.581341825154507,-3.1220149091683833)-- ++(-2.5pt,-2.5pt) -- ++(5.0pt,5.0pt) ++(-5.0pt,0) -- ++(5.0pt,-5.0pt);
		\draw[color=uuuuuu] (6.729358923570788,-2.734904036609518) node {$D$};
		\draw [color=uuuuuu] (1.0600044164045799,-3.0606667157378284)-- ++(-2.5pt,-2.5pt) -- ++(5.0pt,5.0pt) ++(-5.0pt,0) -- ++(5.0pt,-5.0pt);
		\draw[color=uuuuuu] (0.7233589235707943,-2.690904036609518) node {$T$};
		\draw [color=uuuuuu] (6.6334177706062825,1.5648066944998584)-- ++(-2.5pt,-2.5pt) -- ++(5.0pt,5.0pt) ++(-5.0pt,0) -- ++(5.0pt,-5.0pt);
		\draw [color=uuuuuu] (3.8617737962416823,1.595602738659465)-- ++(-2.5pt,-2.5pt) -- ++(5.0pt,5.0pt) ++(-5.0pt,0) -- ++(5.0pt,-5.0pt);
		\draw[color=uuuuuu] (4.023358923570791,1.9950959633904781) node {$H$};
		\draw [color=uuuuuu] (7.415842784909827,1.556113083229819)-- ++(-2.5pt,-2.5pt) -- ++(5.0pt,5.0pt) ++(-5.0pt,0) -- ++(5.0pt,-5.0pt);
		\draw[color=uuuuuu] (7.719358923570788,2.1930959633904776) node {$G$};
		\draw [color=uuuuuu] (5.638808290575755,1.575857910944642)-- ++(-2.5pt,-2.5pt) -- ++(5.0pt,5.0pt) ++(-5.0pt,0) -- ++(5.0pt,-5.0pt);
		\draw[color=uuuuuu] (5.783358923570789,1.9730959633904783) node {$M$};
		\draw [color=uuuuuu] (4.939888935469334,0.8681279248895978)-- ++(-2.5pt,-2.5pt) -- ++(5.0pt,5.0pt) ++(-5.0pt,0) -- ++(5.0pt,-5.0pt);
		\draw[color=uuuuuu] (4.595358923570791,1.3790959633904787) node {$T_1$};
		\end{scriptsize}
		\end{tikzpicture}
		\end{center}			
		La solution utilise deux résultats de géométrie projective avancés : 
		
		i) Pour un point et un cercle donnés, les deux points de contact avec les tangentes par ce point et deux points sur une sécantes sont en division harmonique
		
		ii) Le birapport de quatre points sur un cercle est égal au birapport des tangentes en ces points
		
		iii) Soient $A,B$ des points et $M$ le milieu de $[AB]$. Alors $A,B,M,\infty$ sont harmoniques. Réciproquement, si $A,B,M,\infty$ sont harmoniques, alors $M$ le milieu de $[AB]$.
		
		Début de la solution :
		
		Soit $T_1$ le point de contact de la tangente (autre que $(TC)$) au cercle inscrit. Et soit $M_1$ le point d'intersection de $TT_1$ avec $(GH)$. Montrons que $M_1$ est le mileu de $(GH)$.
				
		D'après i) $T_1 F E D$ sont en division harmonique
		
		D'après ii) $$-1 =b_{T_1,D,F,E}=b_{T_1M_1,DB,FH,EG}=b_{M_1,\infty,H,G}$$
		
		Donc, d'après iii), $M_1$ est bien le milieu de $[HG]$.
\end{sol}

\begin{sol}[130](Résolu par Thibaut Maron)

Soit $x_1,x_2,\dots,x_{13}$ $13$ r\'eels deux \`a deux distincts.\\
La fonction tangente réalise une surjection (en fait une bijection) de $]-\frac{\pi}{2},\frac{\pi}{2}[$ dans $\mathbb{R}$, donc il existe $13$ réels $a_1,a_2,\dots,a_{13}$ appartenant tous \`a $]-\frac{\pi}{2},\frac{\pi}{2}[$ (intervalle de largeur $\pi$),
vérifiant $\forall i \in [[1,13]], \tan(a_i)=x_i$. \\
Or, par le principe des tiroirs, il existe $i$ et $j$ distincts tels que $ 0 \leq a_i-a_j \leq \frac{\pi}{12}$
Or la fonction tangente est croissante sur $]-\frac{\pi}{2},\frac{\pi}{2}[$, donc on peut composer l'in\'egalit\'e pr\'ec\'edente par tangente, \\
on obtient ainsi $\tan(0) \leq \tan(a_i-a_j) \leq \tan(\frac{\pi}{12})$, \\
en appliquant une formule de trigonométrie, \\
on a donc $0 \leq \frac{\tan(a_i)-\tan(a_j)}{1+\tan(a_i)\tan(a_j)} \leq 2-\sqrt{3} $ \\
Or $\forall i \in [[1,13]], \tan(a_i)=x_i$, \\
ainsi $0 \leq \frac{x_i-x_j}{1+x_i x_j} \leq 2-\sqrt{3}$,
ce qui conclut.
\end{sol}

\begin{sol}[59](R\'esolu par Th\'eodore Fougereux et Timoth\'ee Roquet)

		Soit, pour $0 < x < \frac{\pi}{2}$, $f(x)=\left(1+\frac{1}{(\cos x)^{10}}\right)\left(1+\frac{1}{(\sin x)^{10}}\right)$.
		$f$ est d\'erivable et sa d\'eriv\'ee est \\ $f':~x \longmapsto \frac{10\sin(x)}{(\cos(x))^{11}}\left(1+\frac{1}{(\sin(x))^{10}}\right) - \frac{10 \cos(x)}{(\sin(x))^{11}}\left(1+\frac{1}{(\cos(x))^{10}}\right)$. \\
		Si $0 < x < \frac{\pi}{2}$, $f'(x) = \frac{10}{(\sin(x))^{11}(\cos(x))^{11}}\left((\sin(x))^{12}+(\sin(x))^2 - ((\cos(x))^{12}+(\cos(x))^2)\right)$, donc $f'(x)$ a le signe de $g(\sin(x))-g(\cos(x))$, o\`u $g:~x \longmapsto x^{12}+x^2$ qui est strictement croissante sur $\mathbb{R}^+$, et par cons\'equent $f$ d\'ecro\^it avant $\frac{\pi}{4}$ et cro\^it apr\`es. Par cons\'equent, $f$ est minimale en $\frac{\pi}{4}$ et on calcule $f\left(\frac{\pi}{4}\right)=1089$, ce qui conclut.\\
		
		\textit{Une autre preuve est possible}: pour $x > 0$, on note $f(x)=\ln\left(1+\frac{1}{x^5}\right)$, $f$ est d\'erivable et pour $x > 0$, $f'(x)=\frac{\frac{-5}{x^6}}{1+\frac{1}{x^5}}=\frac{-5}{x^6+x}$, et en cons\'equence $f'$ est croissante, donc $f$ est convexe. On a, pour $0 < x < \frac{\pi}{2}$, 
		\[\ln\left(\left(1+\frac{1}{(\cos(x))^{10}}\right)\left(1+\frac{1}{(\sin(x))^{10}}\right)\right) = f((\cos(x))^2)+f((\cos(x))^2) \geq 2f\left(\frac{(\cos(x))^2+(\sin (x))^2}{2}\right) = 2 \ln(33) = \ln (1089).\]
\end{sol}

\begin{sol}[64](R\'esolu par Humbert Tristan)

		Si $p!^k | (p^2)!$, $k*v_p(p!) = v_p(p!^k) \leq v_p((p^2)!)$. 
		Or, la formule de Legendre donne $v_p(p!)=p$, et $v_p((p^2)!)=p+1$. En cons\'equence, si $p!^k | p^2!$, $k \leq p+1$. \\
		On va montrer que $p!^{p+1} | p^2!$. Soit $q$ un premier divisant $p!$ et distinct de $p$ (le r\'esultat du paragraphe est, on l'a vu, vrai pour $p$), soit $i \in \N$, \'ecrivons $p=q^i*a+r$, avec $0 \leq r < q^i$, et $a,r$ entiers. Par primalit\'e de $p$, $r > 0$, on a alors \\ $\left\lfloor \frac{p^2}{q^i}\right\rfloor = \left\lfloor \frac{a^2q^{2i}+2arq^i+r^2}{q^i}\right\rfloor=a^2q^i+2ar+\left\lfloor \frac{r^2}{q^i}\right\rfloor \geq a(aq^i+r+r) \geq (p+1) \left\lfloor \frac{p}{q^i}\right\rfloor$. En sommant, on obtient, pour tout premier $q < p$, $(p+1)v_q(p!) \leq v_q(p^2!)$. \\
		On a alors $p!^{p+1} | p^2!$, et tel n'est pas le cas pour $p!^{p+2}$. 
\end{sol}

\begin{sol}[133](Résolu par Yakob Kahane)

		Si on a deux points sur un cercle, alors le centre dudit cercle se trouve sur la m\'ediatrice des deux points. Si lesdits deux points sont \`a coordonn\'ees rationnelles, leur m\'ediatrice poss\`ede une \'equation \`a coefficients rationnels. \\
			On suppose d\`es lors qu'il existe $A$, $B$, $C$ trois points \`a coordonn\'ees rationnelles sur un m\^eme cercle $\Gamma$ dont le centre n'est pas \`a coordonn\'ees rationnelles. On note $L_1$, $L_2$ des \'equations respectives des m\'ediatrices de $[AB]$ et $[AC]$ \`a coefficients rationnels:
			\[L_1: a_1x+b_1y=c_1\]
			\[L_2: a_2x+b_2y=c_2\]
			Leur point d'intersection est la solution d'un syst\`eme de deux \'equations \`a deux inconnues \`a coefficients rationnels; le calcul de la solution (ie du point d'intersection) se fait en restant dans $\Q$; or ce point $O$ d'intersection est le centre d'un cercle circonscrit \`a $ABC$ donc de $\Gamma$, ce qui contredit l'hypoth\`ese. \\
			Il reste \`a trouver deux points rationnels sur un cercle de centre de coordonn\'ees irrationnelles. Le lecteur v\'erifiera qu'un cercle de centre $O(\sqrt{2};1-\sqrt{2})$ passe par $A(0;0)$ et $B(1;1)$,ce qui conclut.  
\end{sol}

\begin{sol}[13](R\'esolu par Jules Bouton)

		\definecolor{uuuuuu}{rgb}{0.26666666666666666,0.26666666666666666,0.26666666666666666}
\definecolor{zzttqq}{rgb}{0.6,0.2,0.}
\definecolor{qqqqff}{rgb}{0.,0.,1.}
\begin{tikzpicture}[line cap=round,line join=round,>=triangle 45,x=1.0cm,y=1.0cm]
\clip(-4.3,-2.56) rectangle (7.06,6.3);
\fill[color=zzttqq,fill=zzttqq,fill opacity=0.1] (-0.78,2.78) -- (3.64,-0.96) -- (-2.06,-1.18) -- cycle;
\draw [color=zzttqq] (-0.78,2.78)-- (3.64,-0.96);
\draw [color=zzttqq] (3.64,-0.96)-- (-2.06,-1.18);
\draw [color=zzttqq] (-2.06,-1.18)-- (-0.78,2.78);
\draw [domain=-2.06:7.0600000000000005] plot(\x,{(--1.6788--2.9*\x)/3.64});
\begin{scriptsize}
\draw [fill=qqqqff] (-0.78,2.78) circle (2.5pt);
\draw[color=qqqqff] (-0.64,3.14) node {$A$};
\draw [fill=qqqqff] (3.64,-0.96) circle (2.5pt);
\draw[color=qqqqff] (3.78,-0.6) node {$B$};
\draw [fill=qqqqff] (-2.06,-1.18) circle (2.5pt);
\draw[color=qqqqff] (-1.92,-0.82) node {$C$};
\draw [fill=uuuuuu] (1.0096989966555185,1.2656393105222536) circle (1.5pt);
\draw[color=uuuuuu] (1.14,1.54) node {$I$};
\end{scriptsize}
\end{tikzpicture}

		\begin{itemize}
				\item[$\Rightarrow$] La loi des sinus donne $\frac{AI}{\sin(\widehat{ACI})}=\frac{AC}{\sin(\widehat{AIC})}$ et $\frac{BI}{\sin(\widehat{ICB})}=\frac{BC}{\sin(\widehat{CIB})}$, ce qui donne $\frac{AI}{AC}=\frac{\sin(\widehat{ACI})}{\sin(\widehat{AIC})}$ et $\frac{BI}{BC}=\frac{\sin(\widehat{ICB})}{\sin(\widehat{CIB})}$. Comme $\widehat{AIC}$ et $\widehat{CIB}$ sont suppl\'ementaires, ils ont m\^eme sinus. Comme $\widehat{ACI}=\widehat{ICB}$ par hypoth\`ese, on d\'eduit$\frac{BI}{BC}=\frac{AI}{AC}$, c'est-\`a-dire $\frac{CB}{CA}=\frac{IB}{IA}$. \\
				\item[$\Leftarrow$] La loi des sinus donne $\frac{AI}{\sin(\widehat{ACI})}=\frac{AC}{\sin(\widehat{AIC})}$ et $\frac{BI}{\sin(\widehat{ICB})}=\frac{BC}{\sin(\widehat{CIB})}$, ce qui donne $\frac{AI}{AC}=\frac{\sin(\widehat{ACI})}{\sin(\widehat{AIC})}$ et $\frac{BI}{BC}=\frac{\sin(\widehat{ICB})}{\sin(\widehat{CIB})}$. De $\frac{CA}{CB} = \frac{IA}{IB}$ vient $\frac{AI}{AC}=\frac{BI}{BC}$, et en cons\'equence $\frac{\sin(\widehat{ACI})}{\sin(\widehat{AIC})}=\frac{\sin(\widehat{BCI})}{\sin(\widehat{BIC})}$. Comme $\widehat{ACI}$ et $\widehat{BIC}$ sont suppl\'ementaires, ils ont m\^eme sinus.
				On d\'eduit \\ $\sin(\widehat{ACI})=\sin(\widehat{ICB})$, donc les deux angles sont \'egaux (ils ne peuvent pas \^etre suppl\'ementaires) et $(CI)$ est la bissectrice int\'erieure de $\widehat{ACB}$. 
		\end{itemize}~\\
		\textit{On peut aussi prouver le r\'esultat comme suit:} soient $ABC$ un triangle acutangle, $I$ un point du segment $[AB]$, $a = \frac{AI}{AC}$, $b = \frac{BI}{BC}$. De la loi des sinus on d\'eduit $\frac{AI}{\sin(\widehat{ACI})}=\frac{AC}{\sin(\widehat{AIC})}$ et \\ $\frac{BI}{\sin(\widehat{ICB})}=\frac{BC}{\sin(\widehat{CIB})}$, c'est-\`a-dire $a = \frac{\sin(\widehat{ACI})}{\sin(\widehat{AIC})}$ et $b=\frac{\sin(\widehat{BCI})}{\sin(\widehat{BIC})}$. Comme $\widehat{ACI}$ et $\widehat{BIC}$ sont suppl\'ementaires, ils ont m\^eme sinus, d'o\`u $\frac{b}{a}=\frac{\sin(\widehat{ACI})}{\sin(\widehat{ICB})}$. \\
		Finalement, $\frac{CA}{CB}=\frac{IA}{IB} \Leftrightarrow a = b \Leftrightarrow \widehat{ACI}=\widehat{ICB}$, le dernier point \'etant \'equivalent \`a "$(CI)$ est la bissectrice int\'erieure de $\widehat{ACB}$"'.
\end{sol}

\begin{sol}[1](Résolu par ....)

Lorem Ipsum

\end{sol}
