\subsubsection{Principe de r\'ecurrence}


\paragraph{Introduction.}
La r\'ecurrence est un principe de raisonnement math\'ematique, utile dans de nombreux domaines, notamment en combinatoire et en arithm\'etique.

Elle s'appuie sur l'id\'ee que l'on peut parcourir l'ensemble des entiers naturels $\N$ de la mani\`ere suivante : 
on part de $0$, puis on va en $0+1=1$, puis en $1+1=2$, puis en $2+1=3$... 
et ainsi de suite : \`a  chaque fois qu'on se trouve en un entier donn\'e $n$ (autrement dit, \`a  chaque fois qu'on se place au rang $n$), on peut aller \`a  l'entier suivant, qui sera $n+1$.

\paragraph{D\'efinitions et notations.}
Ce qui est alors int\'eressant, c'est qu'on peut d\'efinir des objets qui d\'ependent de notre entier $n$. Par exemple, on peut proc\'eder ainsi : \`a  chaque fois qu'on se place en un entier $n$, on choisit un nombre et on dit qu'on l'associe \`a  notre entier $n$. Pour bien souligner cette association entre le nombre choisi et l'entier $n$ sur lequel on \'etait plac\'e lorsqu'on a choisi le nombre, on note $u_n$ le nombre associ\'e \`a  l'entier $n$.

Quand on regarde $u_0$, puis $u_1$, puis $u_2$,\ldots, et ainsi de suite tous les $u_n$ que l'on a d\'efinis, on d\'efinit en fait une suite de nombres, not\'ee en l'occurrence $(u_n)$. Il ne faut pas confondre la suite, not\'ee $(u_n)$, qui est la succession de tous les nombres, et le terme de la suite au rang $n$, not\'e $u_n$, qui n'est qu'un seul nombre.

Enfin, on peut aussi proc\'eder ainsi : \`a  chaque fois que l'on se place en un entier $n$, on choisit non pas un nombre, mais une propri\'et\'e ; autrement dit, on associe \`a  chaque entier $n$ une propri\'et\'e. On note alors ${\cal P}_n$ la propri\'et\'e associ\'ee \`a  l'entier $n$.

Une telle propri\'et\'e peut \^etre {\it a priori} vraie ou fausse. Dans la pratique, on s'int\'eresse surtout aux propri\'et\'es vraies et g\'en\'erales : on veut souvent que la propri\'et\'e ${\cal P}_n$ reste vraie lorsque l'on remplace $n$ par n'importe quel entier naturel (par exemple, ${\cal P}_n$ : "$n$ est pair" est fausse pour $n=3$ mais la propri\'et\'e ${\cal P}_n$ : "Si $n$ est pair, alors $n^2$ est pair et si $n$ est impair, alors $n^2$ est impair" est vraie pour tout entier $n$).

\paragraph{Principe de r\'ecurrence.}
Consid\'erons une propri\'et\'e {\it d\'ependant} d'un entier $n$, not\'ee ${\cal P}_n$.

Pour montrer que ${\cal P}_n$ est vraie {\it pour tout} $n$ , on peut s'appuyer sur la description pr\'ec\'edente de $\N$. Ainsi, on peut repr\'esenter la situation par un escalier, dont la $n$-i\`eme marche correspond
\`a  la propri\'et\'e ${\cal P}_n$.

\begin{itemize}

\item[$\triangleright$] On commence par v\'erifier que la propri\'et\'e $P_0$ est vraie (l'escalier repose sur un sol ferme !).
\item[$\triangleright$] Ensuite, on d\'emontre que : si la propri\'et\'e est vraie pour un entier donn\'e $n$, 
alors elle est encore vraie pour l'entier suivant $n+1$ (d'une marche \`a  l'autre, il n'y a qu'un pas\ldots \`a  franchir !). Utiliser le principe de r\'ecurrence pr\'esente donc l'int\'er\^et de pouvoir utiliser l'information fournie par ${\cal P}_n$ pour prouver ${\cal P}_{n+1}$.

\end{itemize}


\begin{center}

\definecolor{aqaqaq}{rgb}{0.6274509803921569,0.6274509803921569,0.6274509803921569}
\begin{tikzpicture}[line cap=round,line join=round,>=triangle 45,x=1.0cm,y=1.0cm]
\clip(-1.0,-1.0) rectangle (6.0,6.1);
\fill[color=aqaqaq,fill=aqaqaq,fill opacity=1.0] (-1.0,0.0) -- (-1.0,-0.2) -- (6.0,-0.2) -- (6.0,0.0) -- cycle;
\draw [line width=1.2pt] (-1.0,0.0)-- (0.0,0.0);
\draw [line width=1.2pt] (0.0,0.0)-- (0.0,1.0);
\draw [line width=1.2pt] (0.0,1.0)-- (1.0,1.0);
\draw [line width=1.2pt] (1.0,1.0)-- (1.0,2.0);
\draw [line width=1.2pt] (1.0,2.0)-- (2.0,2.0);
\draw [line width=1.2pt,dotted] (2.0,2.0)-- (2.0,3.0);
\draw [line width=1.2pt,dotted] (2.0,3.0)-- (3.0,3.0);
\draw [line width=1.2pt] (3.0,3.0)-- (3.0,4.0);
\draw [line width=1.2pt] (3.0,4.0)-- (4.0,4.0);
\draw [line width=1.2pt] (4.0,4.0)-- (4.0,5.0);
\draw [line width=1.2pt] (4.0,5.0)-- (5.0,5.0);
\draw [line width=1.2pt,dotted] (5.0,5.0)-- (5.0,6.0);
\draw [line width=1.2pt,dotted] (5.0,6.0)-- (6.0,6.0);
\draw (-0.7466666666666668,-0.1333333333333349) node[anchor=north west] {$\textcolor{blue}{{\cal P}_0}$};
\draw (0.2333333333333349,0.7466666666666668) node[anchor=north west] {${\cal P}_1$};
\draw (1.2333333333333349,1.7466666666666668) node[anchor=north west] {${\cal P}_2$};
\draw (3.2333333333333349,3.7466666666666668) node[anchor=north west] {$\textcolor{blue}{{\cal P}_{n}}$};
\draw (4.1,4.7466666666666668) node[anchor=north west] {$\textcolor{blue}{{\cal P}_{n+1}}$};
\end{tikzpicture}

{\textsc{Figure.--} Escalier de r\'ecurrence}

\end{center}


Plus formellement, un raisonnement par r\'ecurrence se r\'edige ainsi :\\


On consid\`ere, pour $n\in\N$, la propri\'et\'e ${\cal P}_n$ : \og ... \fg.
 
Montrons par r\'ecurrence que ${\cal P}_n$ est vraie pour tout $n\in\N$.

\begin{itemize}

\item[$\triangleright$]
\underline{Initialisation} : 
on v\'erifie que la propri\'et\'e ${\cal P}_0$ est vraie.

\item[$\triangleright$]
\underline{H\'er\'edit\'e} : 
on suppose qu'il existe un entier naturel $n$ tel que la propri\'et\'e ${\cal P}_n$ est vraie (c'est \emph{l'hypoth\`ese de r\'ecurrence}). 
On montre qu'alors la propri\'et\'e ${\cal P}_{n+1}$ est aussi vraie, en utilisant l'hypoth\`ese de r\'ecurrence.

\end{itemize}



\bigskip

\begin{exo}
Soit $n$ un entier naturel. On cherche à exprimer la somme $S_n = 0+1+2+...+n$ autrement (une formule plus simple, qui permette des calculs plus rapides). Certains connaissent d\'ejà la formule $S_n = \frac{n(n+1)}{2}$. Cette formule ne sort pas de nulle part, elle peut se trouver de mani\`ere astucieuse mais pour l'instant contentons-nous de la prouver par r\'ecurrence, avec une r\'edaction la plus propre possible.
\end{exo}

\begin{sol}
Pour $n$ un entier naturel, on note ${\cal P}_n$ la propri\'et\'e : "$S_n =  \frac{n(n+1)}{2}$".

\begin{itemize}

\item[$\triangleright$]
\underline{Initialisation} : 
Pour $n=0$, on a $S_0 =0$ et $\frac{0(0+1)}{2}=0$ donc ${\cal P}_0$ est vraie.

\item[$\triangleright$]
\underline{H\'er\'edit\'e} : 
Soir $n$ un entier naturel tel que la propri\'et\'e ${\cal P}_n$ est vraie. 
On a alors :
\begin{align*}
S_{n+1} &= 0+1+...+n+(n+1)& \\
	     &=S_n + (n+1)& \\
	     &=\frac{n(n+1)}{2} + (n+1) & {\text {(par hypoth\`ese de r\'ecurrence)}}\\
	     &= \frac{n(n+1) + 2(n+1)}{2}& \\
	&= \frac{(n+1)(n+2)}{2}& 
\end{align*}

L'\'egalit\'e $S_{n+1} = \frac{(n+1)(n+2)}{2}$ montre exactement que la propri\'et\'e ${\cal P}_{n+1}$ est vraie.

\item[$\triangleright$]
\underline {Conclusion} :
Grâce au principe de r\'ecurrence, on a montr\'e que l'\'egalit\'e $0+1+ ...+n =  \frac{n(n+1)}{2}$ est vraie pour tout entier $n$.

\end{itemize}
\end{sol}
\begin{exo}

Soit $n\ge 1$ un entier. On trace $n$ cercles dans le plan. Montrer qu'on peut colorier chaque r\'egion du plan ainsi d\'elimit\'ee avec exactement deux couleurs (\textcolor{blue!70}{bleu} et \textcolor{red!80}{rouge} en l'occurrence) de mani\`ere \`a  ce que deux r\'egions s\'epar\'es par un arc de cercle soient toujours de couleur diff\'erente.

\def\firstcircle{(0,0) circle (1.5)}
\def\secondcircle{(2,-1) circle (2)}
\def\thirdcircle{(1.5,0.5) circle (1)}

\begin{center}

\begin{tikzpicture}

    \draw \firstcircle;
    \draw \secondcircle;
    \draw \thirdcircle;
    
    \begin{scope}[even odd rule]
         \clip \secondcircle (-2,-3.5) rectangle (4.5,2); 
         \clip \thirdcircle (-2,-3.5) rectangle (4.5,2);
         \clip \firstcircle (-2,-3.5) rectangle (4.5,2);
     \fill[pattern=north east lines, pattern color=red] (-2,-3.5) rectangle (4.5,2);
     \end{scope}

    \begin{scope}[even odd rule]
         \clip \secondcircle (-2,-3.5) rectangle (4.5,2); 
         \clip \thirdcircle (-2,-3.5) rectangle (4.5,2);
     \fill[pattern=north west lines, pattern color=blue] \firstcircle;
     \end{scope}
     
     \begin{scope}[even odd rule]
         \clip \firstcircle (-2,-3.5) rectangle (4.5,2);
         \clip \thirdcircle (-2,-3.5) rectangle (4.5,2);
     \fill[pattern=north west lines, pattern color=blue] \secondcircle;
     \end{scope}
     
     \begin{scope}[even odd rule]
         \clip \secondcircle (-2,-3.5) rectangle (4.5,2);
         \clip \firstcircle (-2,-3.5) rectangle (4.5,2);
     \fill[pattern=north west lines, pattern color=blue] \thirdcircle;
     \end{scope}
    
    \begin{scope}[even odd rule]
      \clip \thirdcircle;
      \clip \secondcircle  (-2,-3.5) rectangle (4.5,2);
      \fill[pattern=north east lines, pattern color=red] \firstcircle;
    \end{scope}
    
    \begin{scope}[even odd rule]
      \clip \secondcircle;
      \clip \firstcircle  (-2,-3.5) rectangle (4.5,2);
      \fill[pattern=north east lines, pattern color=red] \thirdcircle;
    \end{scope}
    
    \begin{scope}[even odd rule]
      \clip \firstcircle;
      \clip \thirdcircle  (-2,-3.5) rectangle (4.5,2);
      \fill[pattern=north east lines, pattern color=red] \secondcircle;
    \end{scope}

    \begin{scope}
      \clip \firstcircle;
      \clip \secondcircle;
      \fill[pattern=north west lines, pattern color=blue] \thirdcircle;
    \end{scope}
    
\end{tikzpicture}

{\textsc{Figure.--} Exemple de coloriage possible pour $n=3$ cercles}

\end{center}

\end{exo}

\begin{sol}

Si l'on n'a qu'un seul cercle, on colorie l'int\'erieur d'une couleur arbitraire, l'ext\'erieur de l'autre couleur et c'est gagn\'e.

Si on peut toujours colorier $n$ cercles comme voulu, qu'en est-il de $n+1$ cercles ? Pour colorier les r\'egions comme voulu, on oublie tout d'abord un cercle : il en reste alors $n$, et on peut colorier les r\'egions correspondantes comme voulu, par hypoth\`ese de r\'ecurrence. Puis on rajoute le cercle oubli\'e. Il coupe en deux certaines r\'egions colori\'ees : on change alors la couleur de chaque r\'egion coloriage \`a  l'int\'erieur de notre cercle, et on garde la couleur initiale pour les autres r\'egions.

Alors, on a bien colori\'e toutes nos r\'egions comme voulu ! En effet, c'est gagn\'e quand on regarde deux r\'egions s\'epar\'ees par un arc du cercle oubli\'e (gr\^ace au changement qu'on a fait), mais aussi quand on regarde deux r\'egions s\'epar\'ees par un arc d'un autre cercle (par hypoth\`ese de r\'ecurrence).

\end{sol}


\begin{exo}
Pour monter les escaliers, Evariste sait faire des pas de $1$ marche et des pas de $2$ marches. De combien de mani\`eres diff\'erentes peut-il monter un escalier à $n$ marches ?
\end{exo}

\begin{sol}
Pour commencer, notons $u_n$ le nombre cherch\'e. L'\'enonc\'e semble demander une formule explicite pour $u_n$, mais avant cela cherchons ce qu'on appelle une \emph{relation de r\'ecurrence}, c'est-à-dire une expression de $u_n$ en fonction des termes $u_k$ pr\'ec\'edents ($k < n$).

Pour $n\geq 2$ on distingue deux cas : soit Evariste commence par un pas de $1$ marche, dans ce cas il a $u_{n-1}$ mani\`eres de monter les $n-1$ marches restantes ; soit il commence par un pas de $2$ marches et il a $u_{n-2}$ possibilit\'es pour finir l'escalier. Ainsi, on obtient la relation $u_n = u_{n-1}+u_{n-2}$.

Comme cette relation de r\'ecurrence fait intervenir les deux termes pr\'ec\'edant $u_n$, il faut initialiser notre suite avec les deux premiers termes. Quand il y a $0$ ou $1$ marches, Evariste n'a qu'une seule mani\`ere d'avancer. On a donc $u_0=1$ et $u_1=1$.

Il se trouve que la suite ainsi d\'efinie est une suite de r\'ecurrence bien connue : la fameuse suite de Fibonacci ! Je laisse les lecteurs d\'esireux d'en savoir plus se renseigner par eux-m\^emes, \'etablir la formule explicite de $u_n$ ou retrouver la construction du nombre d'or à la r\`egle et au compas...
\end{sol}
